\documentclass{article}
\usepackage{polski}
\usepackage[utf8]{inputenc}

\usepackage{amsthm}
\usepackage{amssymb}
\usepackage{amsmath}
\usepackage{etoolbox}


\theoremstyle{plain}%
\newtheorem{defn}{Definicja}

\theoremstyle{definition}
\newtheorem{zad}{Zadanie}
\AfterEndEnvironment{zad}{\noindent\ignorespaces}

\newtheorem{theo}{Stwierdzenie}

\title{Topologia *}
\author{Michal Zmyslowski}
\date{Listopad 2017}

\begin{document}

\maketitle

\begin{defn}
(Topologie na prostej) W zbiorze liczb rzeczywistych $\mathbb{R}$ definiujemy rodziny podzbiorów $\mathcal{T}_{i}$:\\
1. $\mathcal{T}_{1}=\mathcal{P}(\mathbb{R})$ - topologia dyskretna
\end{defn}

\begin{zad}
Niech $\mathcal{T}_{i}$ będą rodzinami podzbiorów prostej rzeczywistej opisanymi w Definicji 1.\\
a) Sprawdź, że rodziny $\mathcal{T}_{i}$ są topologiami.\\
b) Porównaj topologie $\mathcal{T}_{i}$, rysując diagram inkluzji tych Topologii i zbadaj ich przecięcia.\\
c) Zbadaj, które topologie $\mathcal{T}_{i}$ mają własność Hausdorffa.\\
d) O których parach przestrzeni $(\mathbb{R},\mathcal{T}_{i})$,$(\mathbb{R},\mathcal{T}_{j})$ potrafisz powiedzieć, że są lub nie są homeomorficzne? Narysuj i wypełnij tabelkę.
\end{zad}
b) Ewidentnie $\mathcal{T}_{2}\subset\mathcal{T}_{1}$ i $\mathcal{T}_{3}\subset\mathcal{T}_{1}$. Następnie $[0,1)\in\mathcal{T}_{2}$, ale $[0,1)\not\in\mathcal{T}_{3}$. Podobnie $(0,1]\in\mathcal{T}_{3}$, ale $(0,1]\not\in\mathcal{T}_{2}$. Mamy, że $(0,1)\in\mathcal{T}_{4}$, jak również $(0,1)\in\mathcal{T}_{3}$ i $(0,1)\in\mathcal{T}_{2}$. Jednak $[0,1)\not\in\mathcal{T}_{4}$ i $(0,1]\not\in\mathcal{T}_{4}$. Czyli $\mathcal{T}_{4}\subset\mathcal{T}_{2}$ i $\mathcal{T}_{4}\subset\mathcal{T}_{3}$. Teraz $(-\infty,1)\in\mathcal{T}_{5}$, jak również $(-\infty,1)\in\mathcal{T}_{4}$. Podobnie $(1, \infty)\in\mathcal{T}_{6}$ i $(1, \infty)\in\mathcal{T}_{4}$. Jednak $(0,1)\not\in\mathcal{T}_{5}$ i $(0,1)\not\in\mathcal{T}_{6}$. Tak więc, $\mathcal{T}_{5}\subset\mathcal{T}_{4}$ i $\mathcal{T}_{6}\subset\mathcal{T}_{4}$. Mamy $(-\infty,1)\not\in\mathcal{T}_{7}$ i $(1, \infty)\not\in\mathcal{T}_{7}$. Teraz $\mathbb{R}\setminus\{1\}\in\mathcal{T}_{7}$, ale $\mathbb{R}\setminus\{1\}\not\in\mathcal{T}_{5}$ i $\mathbb{R}\setminus\{1\}\not\in\mathcal{T}_{6}$. Jednak $\mathbb{R}\setminus\{1\}\in\mathcal{T}_{4}$. Czyli $\mathcal{T}_{7}\subset\mathcal{T}_{4}$. Ostatecznie $\mathcal{T}_{8}\subset\mathcal{T}_{5}$, $\mathcal{T}_{8}\subset\mathcal{T}_{6}$,
$\mathcal{T}_{8}\subset\mathcal{T}_{7}$.
c) Weźmy dowolne $x,y\in\mathbb{R}$ takie, że $x<y$. Teraz $\mathcal{T}_{1}$ ma własność Hausdorffa, bo $\{x\},\{y\}\in\mathcal{T}_{1}$. Następnie dobierzmy $s,t,r\in\mathbb{R}$, że $x\in(s,t)$ i $y\in(t,r)$. Teraz  $(s,t)\in\mathcal{T}_{4}$ i $(t,r)\in\mathcal{T}_{4}$. Więc $\mathcal{T}_{2}$, $\mathcal{T}_{3}$ i $\mathcal{T}_{4}$ mają własność Hausdorffa.
Przestrzenie $\mathcal{T}_{i}$ dla $i=5,6,7,8$ nie mają własności Hausdorffa.\\
d) Od razu można powiedzieć, że każda $\mathcal{T}_{i}$ dla $i=1,2,3,4$ nie jest hemeomorficzna z żadną z $\mathcal{T}_{j}$ dla $j=5,6,7,8$, bo wcześniejsze mają własność Hausdorffa, a późniejsze nie. Również $\mathcal{T}_{1}$ nie jest homeomorficzna z $\mathcal{T}_{8}$, bo moc pierwszej jest większa od drugiej, co wiadomo ze Wstępu do Matematyki.
\newpage
\begin{zad}
Niech $d_i$ dla $i = 1,2$ bedą dwoma metrykami w zbiorze X. Następujące warunki są równoważne:\\
1. Topologia wyznaczona przez $d_2$ jest drobniejsza niż wyznaczona przez $d_1$, tzn. $\mathcal{T}(d_1)\subset\mathcal{T}(d_2)$.\\
2. Dla każdej kuli $\mathcal{B}_{d_1}(x,r_1)$ istnieje liczba $r_2>0$ taka, że $\mathcal{B}_{d_2}(x,r_2)\subset\mathcal{B}_{d_1}(x,r_1)$.\\
3. Jeśli ciąg jest zbieżny w metryce $d_2$ to jest zbieżny w metryce $d_1$ do tej samej granicy.\\
$1\implies2$\\
Załóżmy, że $\mathcal{T}(d_1)\subset\mathcal{T}(d_2)$. Niech $\mathcal{B}_{d_1}(x,r_1)\in\mathcal{T}(d_1)$. Z założenia $\mathcal{B}_{d_1}(x,r_1)\in\mathcal{T}(d_2)$. Z definicjii topologii $\mathcal{T}(d_2)$ istnieje $r_2>0$ takie, że\\
$\mathcal{B}_{d_2}(x,r_2)\subset\mathcal{B}_{d_1}(x,r_1)$.\\
$2\implies1$\\
Załóżmy, że  dla każdej kuli $\mathcal{B}_{d_1}(x,r_1)$ istnieje liczba $r_2>0$ taka, że\\ $\mathcal{B}_{d_2}(x,r_2)\subset\mathcal{B}_{d_1}(x,r_1)$. Weźmy dowolny $y_s\in\mathcal{B}_{d_1}(x,r_1)$. Z definicji topologii $\mathcal{T}(d_1)$ istnieje $r>0$ takie, że $\mathcal{B}_{d_1}(y_s,r)\subset\mathcal{B}_{d_1}(x,r_1)$. Z założenia istnieje $r_s>0$ takie, że $\mathcal{B}_{d_2}(y_s,r_s)\subset\mathcal{B}_{d_1}(y_s,r)$. Z tego wynika, że $\bigcup_s \mathcal{B}_{d_2}(y_s,r_s) = \mathcal{B}_{d_1}(x,r_1)\in\mathcal{T}(d_1)$. Z definicji topologii $\bigcup_s \mathcal{B}_{d_2}(y_s,r_s)\in\mathcal{T}(d_2)$. Tak, więc $\mathcal{T}(d_1)\subset\mathcal{T}(d_2)$.\\
$1\implies3$\\
Załóżmy, że $\mathcal{T}(d_1)\subset\mathcal{T}(d_2)$. Niech ciąg $\{x_n\}_{n=1}^\infty$ będzie zbieżny do $x$ w metryce $d_2$. Załóżmy, że $\{x_n\}_{n=1}^\infty$ nie jest zbieżny do $x$ w metryce $d_1$. Z tego wynika, że istnieje $\epsilon>0$, taki, że dla każdego $n_{\epsilon}$ istnieje $n>n_{\epsilon}$, że $x_n\not\in\mathcal{B}_{d_1}(x,\epsilon)$. Z założenia $\mathcal{T}(d_1)\subset\mathcal{T}(d_2)$ wynika jednak, że istnieje liczba $r>0$ taka, że $\mathcal{B}_{d_2}(x,r)\subset\mathcal{B}_{d_1}(x,\epsilon)$. Jako, że $\{x_n\}_{n=1}^\infty$ jest zbieżny do $x$ w metryce $d_2$, to z definicji zbieżności prawie wszystkie wyrazy $\{x_n\}_{n=1}^\infty$ znajdują się w $\mathcal{B}_{d_2}(x,r)$, co jest sprzeczne z faktem, że $x_n\not\in\mathcal{B}_{d_1}(x,\epsilon)$. Z tego wynika, że $\{x_n\}_{n=1}^\infty$ jest zbieżny do $x$ w metryce $d_1$.\\
$3\implies1$\\
Załóżmy, że jeśli ciąg jest zbieżny w metryce $d_2$ to jest zbieżny w metryce $d_1$. Zawieranie się topologii jest równoważne zawieraniu się rodziny zbiorów domkniętych, tzn. $\mathcal{F}_{\mathcal{T}(d_1)}\subset\mathcal{F}_{\mathcal{T}(d_2)}\Longleftrightarrow\mathcal{T}(d_1)\subset\mathcal{T}(d_2)$. Dowiedzimy tego faktu w "prawą" stronę. Niech $A\in\mathcal{F}_{\mathcal{T}(d_1)}$. Jako, że $A$ jest zbiorem domkniętym to $X\setminus A$ jest zbiorem otwartym, więc $X\setminus A\in\mathcal{T}(d_1)$. Z założenia wynika, że $A\in\mathcal{F}_{\mathcal{T}(d_2)}$, więc również $X\setminus A\in\mathcal{T}(d_2)$. Niech $M\in\mathcal{F}_{\mathcal{T}(d_1)}$. Z definicji domknięcia dostajemy, że $M\subset cl_{\mathcal{T}(d_2)}(M)$. Jeżeli $x\in cl_{\mathcal{T}(d_2)}(M)$, to istnieje $\{x_n\}_{n=1}^\infty \subset M$, taki, że $d_2(x_n,x)\to 0$. Z założenia dostajemy, że $d_1(x_n,x)\to 0$. Z tego wynika, że $x\in cl_{\mathcal{T}(d_1)}(M)$. M jest domknięty w $\mathcal{T}(d_1)$, czyli $M=cl_{\mathcal{T}(d_1)}(M)$. A z tego mamy, że $x\in M$, co daje
$cl_{\mathcal{T}(d_2)}(M)\subset M$. Wtedy $cl_{\mathcal{T}(d_2)}(M)=M$, a z tego wynika, że $M\in \mathcal{F}_{\mathcal{T}(d_2)}$, czyli  $\mathcal{F}_{\mathcal{T}(d_1)}\subset \mathcal{F}_{\mathcal{T}(d_2)}$. Tak więc, na mocy faktu, który wcześniej udowodniliśmy $\mathcal{T}(d_1)\subset\mathcal{T}(d_2)$.\\
\end{zad}
\begin{defn}
Niech $(X,\mathcal{T})$ będzie przestrzenią topologiczną. Przez $C(X)$ oznaczamy zbiór
funkcji ciągłych $f:(X,\mathcal{T}) \to (\mathbb{R},\mathcal{T}_{e})$, a przez $C_{b}(X)$ jego zbiór składający się z funkcji ograniczonych. Dla dowolnej funkcji $f \in C_{b}(X)$ definiujemy $\|f\|_{sup}:=sup\{|f(x)|:x \in X\}$ oraz $\|f\|_{L^1}:=\int_0^1 |f(t)| dt$.
\end{defn}
\begin{zad}
Porównać topologię wyzanczoną przez normę $\|f\|_{sup}$ z topologią wyznaczoną przez normę $\|f\|_{L^1}$.
\end{zad}
Niech
\[f_{n}(x):= \begin{cases} 
1-nx\ ,\ x \in [0,\frac{1}{n}]\\
0\ ,\ x \in (\frac{1}{n},1)
\end{cases}\] Wtedy  $f_{n} \in C_{b}([0,1])$ i  $\|f_{n}(x)\|_{L^1}=\frac{1}{2n} \to 0$, ale
$\|f_{n}(x)\|_{sup}=1 \not\to 0$.\\
Rozważmy $g_{n} \in C_{b}([0,1])$ taką, że $\|g_{n}(x)\|_{sup} \to 0$.
Wtedy $\|g_{n}(x)\|_{L^1}\leq$\\ $sup(|g_{n}(x)|)(1-0)\to 0$.\\
Niech $d_{sup}(f,g)=\|f-g\|_{sup}$ i $d_{L^1}(f,g)=\|f-g\|_{L^1}$.
Rozpatrzmy $(C_{b}(X), d_{sup})$  i $(C_{b}(X), d_{L^1})$. Jeżeli $f_n$ jest zbieżny do f w przestrzeni $(C_{b}(X), d_{sup})$, tzn. $d_{sup}(f_n,f)\to0$, to z wcześniejszych obserwacji $d_{L^1}(f_n,f)\to0$, czyli jest zbieżny, również do f, w $(C_{b}(X), d_{L^1})$. Z poprzedniego zadania wiemy już, że wtedy $\mathcal{T}(d_{L^1})\subset\mathcal{T}(d_{sup})$.
\end{document}
